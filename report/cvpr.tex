% This version of CVPR template is provided by Ming-Ming Cheng.
% Please leave an issue if you found a bug:
% https://github.com/MCG-NKU/CVPR_Template.

%\documentclass[review]{cvpr}
\documentclass[final]{cvpr}

\usepackage{times}
\usepackage{epsfig}
\usepackage{graphicx}
\usepackage{amsmath}
\usepackage{amssymb}

% Include other packages here, before hyperref.

% If you comment hyperref and then uncomment it, you should delete
% egpaper.aux before re-running latex.  (Or just hit 'q' on the first latex
% run, let it finish, and you should be clear).
\usepackage[pagebackref=true,breaklinks=true,colorlinks,bookmarks=false]{hyperref}


\def\cvprPaperID{****} % *** Enter the CVPR Paper ID here
\def\confYear{CVPR 2022}
%\setcounter{page}{4321} % For final version only


\begin{document}

%%%%%%%%% TITLE
\title{Camera Pose Estimation using Regression Forest and RANSAC Optimization}

\author{Falk Ebert\\
\tt 4018276\\
{\tt\small f.ebert@stud.uni-heidelberg.de}
% For a paper whose authors are all at the same institution,
% omit the following lines up until the closing ``}''.
% Additional authors and addresses can be added with ``\and'',
% just like the second author.
% To save space, use either the email address or home page, not both
\and
Jason Pyanowski\\
\tt 3663907\\
{\tt\small j.pyanowski@stud.uni-heidelberg.de}
\and
Marven Hinze\\
\tt 3664283\\
{\tt\small m.hinze@stud.uni-heidelberg.de}
\and
Nadine Theissen\\
\tt 3475402\\
{\tt\small n.theissen@stud.uni-heidelberg.de}
}

\maketitle


%%%%%%%%% ABSTRACT
\begin{abstract}
   
\end{abstract}

%%%%%%%%% BODY TEXT
\section{Introduction}
cite it~\cite{shotton2013}

% What are we doing in this project?
%	-> Scene coordinate regression
% 	-> Camera pose estimation
% Why is this important?

% What data is given
% 	-> Image data including depth, camera poses

%-------------------------------------------------------------------------
\subsection{Related Work}


\section{Methods}
%-------------------------------------------------------------------------
This section gives an overview of the methods used in this project. We will discuss
the concept of regression forests including feature extraction and the RANSAC algorithm
used for estimating the camera pose from the image data. We will not discuss all aspects
in full detail but only provide further explanations where we find it relevant
for the presentation of our work and results.

\subsection{Data Preperation}
%-------------------------------------------------------------------------
To train the regression forest the 7-scenes datset~\cite{glocker2013} is utilized. 
Each scene consists of multiple image sequences that cover the are of interest. Using a RGB-D 
Kinect camera, depth information for each pixel is extracted. The resolution of the $24$ 
bit RGB-images is $640\times480$ pixels and the corresponding $16$ bit depth map is given 
in millimeters. For each image a $4\times4$ ground truth camera pose in homogenous coordinates 
is provided, which allows to validate the estimated camera pose matrix for a given image. 

The datasets are split randonly into train and test according to~\cite[Table 1]{shotton2013}. 

% WIP FALK:
Our data consists of a number of samples $\{(\textbf{p}_i, \textbf{w}_i)\}$ with
$\textbf{p}_i = (x_l, y_l)_i$ and $\textbf{w}_i = (x_s, y_s, z_s)_i$ where the subscripts
$l$ and $s$ denote image pixel coordinates and scene coordinates respectively.\\

\subsection{Regression Forest}
%-------------------------------------------------------------------------

In this project we use a regression forest to predict scene coordinates for a given
sample image coordinate as suggested in \cite{shotton2013}. In this section we will
give some background on the concept of regression forests, the pixel coordinate
labeling and the image-feature extraction on which the forests base their predictions.\\

\subsubsection{Decision Trees}
%-------------------------------------------------------------------------
A regression forest consists of a number $N$ of regression trees, which each consist of
of root node and it's children. We consider the special case of binary decision trees
where each node (unless it is a leaf node) has a left and a right child. Each node $n$
stores a set of parameters $\theta_n$ which are used in the calculation of a
feature response function $f_{\theta}(\textbf{p}) \in \{1, 0\}$ which determines if the input
data point \textbf{p} is branched to the left or right child node. To every leaf node
(i.e. a node without children) we assign a response value $\textbf{w}_n$ representing
the tree's prediction for a data point \textbf{p} that has reached this node. In our
specific application the input data points \textbf{p} are 2D pixel coordinates and the
responses \textbf{w} are 3D world coordinates.\\

The process of training the tree involves finding the set of parameters
$\{\theta_n | \, \forall \, \text{nodes} \, n\}$ which results in the best predictions
for previously unseen input data points \textbf{p}. However, this final goal cannot
be optimized towards directly during training, as this would entail optimization in the
very large space of all possible parameters for all tree-nodes simultaneously. Therefore
it is much more efficient to optimize the parameters one node at a time using a
proxy-objective $Q(S_{\text{tot}},\, S_{\text{left}}(\theta),\, S_{\text{right}}(\theta))$
which ideally leads to a similar result.\\

Let $S_{\text{tot},n} = \{ (\textbf{p}_i, \textbf{w}_i) \, | \, i \in |S_n|\}$ denote the
input data and target response for a given node $n$. According to its parameters the
node splits this set into the subsets
\begin{equation}
\begin{split}
	S_{\text{left},n}(\theta_n) = \{(\textbf{p}_i, \textbf{w}_i) \,|\, f_{\theta_n}(\textbf{p}_i) = 0\} \\
	S_{\text{right},n}(\theta_n) = \{(\textbf{p}_i, \textbf{w}_i) \,|\, f_{\theta_n}(\textbf{p}_i) = 1\}
\end{split}
\end{equation}
corresponding to left and right child node respectively. The objective function $Q$ is
then used to assign a score to the split resulting from the parameters $\theta_n$ at
this node.\\

The objective function used here optimizes for a reduction in variance of the target 
responses (i.e. "wants the tree to group together points similar scene coordinates").
Defining $W_n = \{ \textbf{w}_i \, | \, (\textbf{p}_i, \textbf{w}_i) \in S_n \}$ this
objective can be mathematically expressed as

\begin{equation}
	Q(W_\text{tot}, \theta) =
		\text{Var}(W_\text{tot}) - \sum_{d\,\in\{\text{left,\,right}\}}
			\dfrac{|W_d(\theta)|}{|W_\text{tot}|} \text{Var}(W_d(\theta))
\end{equation}
\begin{equation}
	\text{with} \hspace{5mm} \text{Var}(W) = |W|^{-1} \sum_{\textbf{w}\in W} ||\textbf{w} - \bar{\textbf{w}}||_2^2
\end{equation}


\subsubsection{Image Features}
%-------------------------------------------------------------------------
In order to use a decision tree to infer scene coordinates from the pixel coordinates,
it is necessary to calculate features associated with a given pixel coordinate from the
image data. 


%-------------------------------------------------------------------------
\subsection{RANSAC Optimization}

%-------------------------------------------------------------------------
\section{Experiment Evaluation}
To evaluate the estimated camera poses, the translational and angular error with respect to the 
ground truth camera poses is measured. In particular, a translational error of at 
most $5$cm and angular error of at most $5^{\circ}$ is considered to be a correctly predicted pose. 

Let the estimated pose matrix $H_{est}$ and the ground truth $H_{gt}$ be $4 \times 4$ matrices
in homogeneous coordinates. Then the translational error $\varepsilon_t$ is obtained by using the 
$L_2$-norm of the translation vectors $T_{est}, T_{gt} \in \mathbb{R}^3$ as 
\begin{align}
    \varepsilon_t &= \sqrt{\sum_{x,y,z}(T_{est} - T_{gt})^2}.
\end{align}

To compare two rotational poses, the angle of the difference rotation matrix $R$ is utilized. 
Denote the whole $3\times3$ upper-left matrix of $H_{est}$ and $H_{gt}$ as $R_{est}$ and 
$R_{gt}$ repectively. They refer to the rotational poses and the difference rotation can be computed 
by $R = R_{est}^TR_{gt}$. The angle $\varepsilon_r$ between those two rotation matrices is then obtained as
\begin{align}
    \varepsilon_r &= arccos \left( \frac{tr(R)-1}{2} \right)
\end{align}
where $tr(R)$ is the tracce of a matrix.

Based on these metrics the test data set for each scene is used to predict the amount of correctly 
estimated camera poses. These findings are then compared to the results of~\cite{shotton2013}.

%-------------------------------------------------------------------------
\section{Results}


{\small
\bibliographystyle{ieee_fullname}
\bibliography{egbib}
}

\end{document}
